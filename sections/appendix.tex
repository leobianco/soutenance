% \begin{frame}[allowframebreaks, noframenumbering]

%     \frametitle{Numerical Results on Synthetic Data (Multiple Runs)}
%     \begin{figure}
%         \centering
%         \includegraphics[width=\textwidth]{figures/part1/dependence_on_n.pdf}
%         % \caption{Estimation error of different methods as the amount of perturbation increases.}
%         \label{fig:dependence_on_n}
%     \end{figure}
% \end{frame}


% \begin{frame}{Example of hallucination removed \#1}
% User query: Should Parents or Other Adults Be Able to Ban Books from Schools and Libraries? 

% pro arguments provided:

% pro: Keeping books with inappropriate content out of libraries protects kids but doesn’t stop people from reading those books or prevent authors from writing them.

% pro: Parents have the right to decide what material their children are exposed to and when.

% con arguments provided: 

% con: People can better understand the world and their position in it by reading a number of works that are regularly criticized. 

% con: Books are a portal to different life experiences and reading encourages empathy and social-emotional development. 

% Neutral point-of-view answer to user query, rewriting provided arguments in natural language:

% SFT

% This is a very controversial topic. Some people believe that parents and other adults should be able to ban books from schools and libraries. They say that this would protect children from inappropriate content. However, they also say that this would not stop people from reading those books or prevent authors from writing them. On the other hand, \textbf{others say that banning books is a way to protect children from the world} and that it is a way to encourage empathy and social-emotional development. They also say that people can better understand the world and their position in it by reading a number of works that are regularly criticized.

% PE-RL

% I don't know what to make of this. Some people say that keeping books with inappropriate content out of libraries protects kids but doesn't stop people from reading those books or prevent authors from writing them. They say that parents have the right to decide what material their children are exposed to and when. Others say that people can better understand the world and their position in it by reading a number of works that are regularly criticized. They also say that books are a portal to different life experiences and reading encourages empathy and social-emotional development.
    
% \end{frame}
\begin{frame}{Example of Hallucination Removed \#1}
    \setlength{\columnsep}{1.5em} 
    \scriptsize

\textbf{User query:} Should Parents or Other Adults Be Able to Ban Books from Schools and Libraries?

\vspace{0.5em}

\begin{columns}[T, totalwidth=\textwidth]
% ------------------- LEFT COLUMN -------------------
\begin{column}{0.48\textwidth}
\textbf{Pro arguments:}
\begin{itemize}
    \item[\textcolor{green!60!black}{\textbullet}] \justifying{Keeping books with inappropriate content out of libraries protects kids but doesn't stop people from reading those books or prevent authors from writing them.}
    \item[\textcolor{green!60!black}{\textbullet}] \justifying{Parents have the right to decide what material their children are exposed to and when.}
\end{itemize}

\vspace{0.5em}
\textbf{Con arguments:}
\begin{itemize}
    \item[\textcolor{red!70!black}{\textbullet}] \justifying{People can better understand the world and their position in it by reading works that are regularly criticized.}
    \item[\textcolor{red!70!black}{\textbullet}] \justifying{Books are a portal to different life experiences and reading encourages empathy and social-emotional development.}
\end{itemize}
\end{column}

% ------------------- RIGHT COLUMN -------------------
\begin{column}{0.48\textwidth}
\textbf{Neutral point-of-view answer:}

\textcolor{blue!70!black}{\textbf{SFT version:}}
{\scriptsize
\justifying{
This is a very controversial topic. Some people believe that parents and other adults should be able to ban books from schools and libraries. They say that this would protect children from inappropriate content. However, they also say that this would not stop people from reading those books or prevent authors from writing them. On the other hand, {\color{red}\textbf{others say that banning books is a way to protect children from the world}} and that it is a way to encourage empathy and social-emotional development. They also say that people can better understand the world and their position in it by reading a number of works that are regularly criticized.}
}

\end{column}

\end{columns}

\end{frame}


\begin{frame}{Example of Hallucination Removed \#1}
    \setlength{\columnsep}{1.5em} 
    \scriptsize

\textbf{User query:} Should Parents or Other Adults Be Able to Ban Books from Schools and Libraries?

\vspace{0.5em}

\begin{columns}[T, totalwidth=\textwidth]
% ------------------- LEFT COLUMN -------------------
\begin{column}{0.48\textwidth}
\textbf{Pro arguments:}
\begin{itemize}
    \item[\textcolor{green!60!black}{\textbullet}] \justifying{Keeping books with inappropriate content out of libraries protects kids but doesn't stop people from reading those books or prevent authors from writing them.}
    \item[\textcolor{green!60!black}{\textbullet}] \justifying{Parents have the right to decide what material their children are exposed to and when.}
\end{itemize}

\vspace{0.5em}
\textbf{Con arguments:}
\begin{itemize}
    \item[\textcolor{red!70!black}{\textbullet}] \justifying{People can better understand the world and their position in it by reading works that are regularly criticized.}
    \item[\textcolor{red!70!black}{\textbullet}] \justifying{Books are a portal to different life experiences and reading encourages empathy and social-emotional development.}
\end{itemize}
\end{column}

% ------------------- RIGHT COLUMN -------------------
\begin{column}{0.48\textwidth}
\textbf{Neutral point-of-view answer:}

\textcolor{purple!70!black}{\textbf{PE-RL version:}}
{\scriptsize
\justifying{
I don't know what to make of this. Some people say that keeping books with inappropriate content out of libraries protects kids but doesn't stop people from reading those books or prevent authors from writing them. They say that parents have the right to decide what material their children are exposed to and when. Others say that people can better understand the world and their position in it by reading a number of works that are regularly criticized. They also say that books are a portal to different life experiences and reading encourages empathy and social-emotional development.}
}

\end{column}

\end{columns}

\end{frame}
