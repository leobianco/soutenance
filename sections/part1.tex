\section{Part I: Contributions to Robust Community Detection}
\subsection{Robust Estimation for the SBM}

% \begin{frame}[plain]
%     \centering
%     \vfill
%     \Huge Part I\\[0.75em]\textbf{Contributions to Robust Community Detection}
%     \vfill
% \end{frame}

\begin{frame}[plain, noframenumbering]{Overview}
    \setlength{\parskip}{1em}
    \tableofcontents[currentsection]
\end{frame}

\setcounter{framenumber}{0}

\begin{frame}{Motivation}
    \begin{columns}
        \begin{column}{0.5\linewidth}
            \begin{itemize}
                \item \textit{Adjacency matrix}: symmetric \(A \in {\{0, 1\}}^{n \times n}\)
                \item[{\color{white}\ding{118}}] {\color{white}\textit{Community detection}~\citep{abbe2023communitydetectionstochasticblock}: group similar nodes, sensitive to \textit{outliers}}
                \item[{\color{white}\ding{118}}] {\color{white}\textit{Robust algorithm}: accurate results despite outliers}
            \end{itemize}
        \end{column}
        \begin{column}{0.5\linewidth}
            \begin{figure}
                \centering
                \includegraphics[width=\textwidth]{figures/part1/jazz_graph.png}
                \caption*{\scriptsize \color{gray} Jazz collaborations in New York, Chicago, and elsewhere~\citep{gleiser2003community}}
            \end{figure}
        \end{column}
    \end{columns}
\end{frame}

\begin{frame}{Motivation}
    \addtocounter{framenumber}{-1}
    \begin{columns}
        \begin{column}{0.5\linewidth}
            \begin{itemize}
                \item \textit{Adjacency matrix}: symmetric \(A \in {\{0, 1\}}^{n \times n}\)
                \item \textit{Community detection}~\citep{abbe2023communitydetectionstochasticblock}: group similar nodes, sensitive to \textit{outliers}
                \item[{\color{white}\ding{118}}] {\color{white}\textit{Robust algorithm}: accurate results despite outliers}
            \end{itemize}
        \end{column}
        \begin{column}{0.5\linewidth}
            \begin{figure}
                \centering
                \includegraphics[width=\textwidth]{figures/part1/jazz_sc_graph.png}
                \caption*{\scriptsize \color{gray}Clustering of the Jazz collaborations}
            \end{figure}
        \end{column}
    \end{columns}
\end{frame}

\begin{frame}{Motivation}
    \addtocounter{framenumber}{-1}
    \begin{columns}
        \begin{column}{0.5\linewidth}
            \begin{itemize}
                \item \textit{Adjacency matrix}: symmetric \(A \in {\{0, 1\}}^{n \times n}\)
                \item \textit{Community detection}~\citep{abbe2023communitydetectionstochasticblock}: group similar nodes, sensitive to \textit{outliers}
                \item \textit{Robust algorithm}: accurate results despite outliers
            \end{itemize}
        \end{column}
        \begin{column}{0.5\linewidth}
            \begin{figure}
                \centering
                \includegraphics[width=\textwidth]{figures/part1/jazz_sc_graph.png}
                \caption*{\scriptsize \color{gray}Clustering of the Jazz collaborations}
            \end{figure}
        \end{column}
    \end{columns}
\end{frame}

% --- METHODOLOGY ---
% \begin{frame}{Graphs}
%     \begin{columns}
%         \column{0.5\linewidth}
%             \textit{Graph:} nodes linked by edges.\\[1em]
%             \hspace{1em}\(G = (V, E)\) \\[1em]
%             \hspace{1em}\(V = \{1, \dotsc, n\}\), \(E \subset V \times V\)\\[1em]
%             \textit{Undirected graph:} \\[1em]
%             \hspace{1em}\((i, j) \in E \Rightarrow (j, i) \in E\)\\[1em]
%             \textit{Adjacency matrix:}\\[1em]
%             \hspace{1em}\(A_{i j} = 
%                 \begin{cases}
%                     \, 1 \quad \text{if } (i, j) \in E \\
%                     \, 0 \quad \text{otherwise}
%                 \end{cases}
%             \)
%         \column{0.5\linewidth}
%             \begin{figure}
%                 \includegraphics[width=\linewidth]{figures/part1/polbooks.png}
%             \end{figure}
%     \end{columns}
% \end{frame}


\begin{frame}{The Stochastic Block Model~\citep{HOLLAND1983109}}
    \begin{columns}
        \column{0.5\linewidth}
            \(Z_i \to \text{community of node } i\) \\[1em]
            \(K \; \to \text{nb. of communities}\) \\[1em]
            \(\textcolor{black}{\pi_{k}} \to \text{size of community } k\) \\[1em]
            \(\textcolor{myorange}{\Gamma_{kl}} \to \text{connectivity } k, l\) \\[1.5em]
            {\color{white}
            \((Z, A) \sim \textup{SBM}_K (\pi, \Gamma)\) 
            \begin{equation*}
                \begin{cases}
                    \mathbb{P}(Z_i = k) = \pi_k \\
                    \mathbb{P}(A_{i j} = 1 \vert Z_i = k, Z_j = l) = \Gamma_{kl}
                \end{cases}
            \end{equation*}
            }
        \column{0.5\linewidth}
            \includegraphics[width=\linewidth]{figures/part1/sbm.png}
    \end{columns}
\end{frame}

\begin{frame}{The Stochastic Block Model~\citep{HOLLAND1983109}}
    \addtocounter{framenumber}{-1}
    \begin{columns}
        \column{0.5\linewidth}
            \(Z_i \to \text{community of node } i\) \\[1em]
            \(K \; \to \text{nb. of communities}\) \\[1em]
            \(\textcolor{black}{\pi_{k}} \to \text{size of community } k\) \\[1em]
            \(\textcolor{myorange}{\Gamma_{kl}} \to \text{connectivity } k, l\) \\[1.5em]
            \((Z, A) \sim \textup{SBM}_{K} (\textcolor{black}{\pi}, \textcolor{myorange}{\Gamma})\) 
            \begin{equation*}
                \begin{cases}
                    \mathbb{P}(Z_i = k) = \textcolor{black}{\pi_k} \\
                    \mathbb{P}(A_{i j} = 1 \vert Z_i = k, Z_j = l) = \textcolor{myorange}{\Gamma_{kl}}
                \end{cases}
            \end{equation*}
        \column{0.5\linewidth}
            \includegraphics[width=\linewidth]{figures/part1/sbm.png}
    \end{columns}
\end{frame}

\begin{frame}{The Corrupted Stochastic Block Model~\citep{liu2022minimax}}
    \begin{columns}
        \column{0.5\linewidth}
            \textit{Adversary} creates outliers:
            \vspace{0.75em}
            \begin{enumerate}
                \item \((Z, A_{\textup{pure}}) \sim \textup{SBM}_{K}(\textcolor{black}{\pi}, \textcolor{myorange}{\Gamma})\)
                \item Adversary arbitrarily changes edges of \(\gamma n\) nodes
                \item Corrupted \(A\) is observed
            \end{enumerate}
            % \begin{equation*}
            %      \; \xrightarrow[\gamma n {\scriptsize \text{ nodes}}]{{\scriptsize \text{Corruption}}} \; A
            % \end{equation*}
            % \\[1em]
            % \hspace{1em} \(Z_i \to \text{community of node } i\) \\[1em]
            % \hspace{1em} \(K \; \to \text{nb. of communities}\) \\[1em]
            % \hspace{1em} \(\textcolor{black}{\pi_{k}} \to \text{size of community } k\) \\[1em]
            % \hspace{1em} \(\textcolor{myorange}{\Gamma_{kl}} \to \text{connectivity } k, l\) \\[1.5em]
            % \((Z, A) \sim \textup{SBM}_{K} (\textcolor{black}{\pi}, \textcolor{myorange}{\Gamma})\) 
            % \begin{equation*}
            %     \begin{cases}
            %         \mathbb{P}(Z_i = k) = \textcolor{black}{\pi_k} \\
            %         \mathbb{P}(A_{i j} = 1 \vert Z_i = k, Z_j = l) = \textcolor{myorange}{\Gamma_{kl}}
            %     \end{cases}
            % \end{equation*}
        \column{0.5\linewidth}
            \includegraphics[width=\linewidth]{figures/part1/sbm_outliers.png}
    \end{columns}
\end{frame}
% --- END OF METHODOLOGY ---

% --- PROBLEM ---
\begin{frame}{Research question}
    \begin{itemize}
        % \item {\color{myred}\textbf{Adversary}}: arbitrary changes on a fraction \(\gamma\) of nodes.
        % \begin{equation*}
        %     (Z, A^\prime) \sim \textup{SBM}_{K}(\textcolor{myblue}{\pi}, \textcolor{myorange}{\Gamma}) \quad \xrightarrow[\gamma n {\scriptsize \text{ nodes}}]{{\scriptsize \text{Corruption}}} \quad A
        % \end{equation*}
        \item \textbf{Problem}: estimate \(\textcolor{myorange}{\Gamma}\) under \textit{worst-case} adversary
        \item For \(K=1\), solved by \citet{pmlr-v178-acharya22a}
    \end{itemize}
    % \pause
    \begin{tcolorbox}[colback=orange!60!white,
                    colframe=orange!60!white]
        \textbf{Research question:} \\[0.5em]
        How to robustly estimate \(\Gamma\) for \(K > 1\)?
    \end{tcolorbox}
\end{frame}
% --- END OF PROBLEM ---

\begin{frame}{Results}
    \begin{itemize}
        \item Idea: find subgraph \(S\) excluding worst outliers
        \begin{equation*}
            \Rightarrow \quad \hat{\Gamma} = \left(\sum_{i \in S_k j \in S_l} A_{i j}\right) / \vert S_k \vert \vert S_l \vert \quad \textup{is a good estimator}
        \end{equation*}
        \pause
        \item \textbf{Contribution}: extend bound in~\citet{pmlr-v178-acharya22a} to \(K > 1\)
    \end{itemize}
    \begin{tcolorbox}[colback=yellow!45!white,
                    colframe=yellow!45!white]
        \textbf{Theorem~\citep{bianco2025subsearchrobustestimationoutlier}.} Let \(S\) be a subgraph clustered into \(S_1, \dotsc, S_K\), \(\Omega_k\) the nodes in community \(k\), \(\mathcal{I}\) the set of inlier nodes.
        Let
        \(\hat{\Gamma} = \left(\sum_{i \in S_k j \in S_l} A_{i j}\right) / \vert S_k \vert \vert S_l \vert\) and \(\hat{Q}(S)_{i j} = \hat{\Gamma}_{S(i) S(j)}\).
        Then,
        \begin{align*}
            \lVert \Gamma - \hat{\Gamma} \rVert_1 &\lesssim \frac{\lVert A_S - \hat{Q}(S) \rVert_{\textup{op}}}{\min_{1 \leq k \leq K} \lvert \Omega_k \cap S_k \cap \mathcal{I} \rvert}
        \end{align*}
    \end{tcolorbox}
\end{frame}

% \begin{frame}{Results}
%     {\color{red}\textbf{Interpretation:}}
% \end{frame}

\begin{frame}{Results}
    \addtocounter{framenumber}{-1}
    \begin{itemize}
        \item Idea: find subgraph \(S\) excluding worst outliers
        \item \textbf{Contribution}: extend bound in~\citet{pmlr-v178-acharya22a} to \(K > 1\)
        \item \textbf{Contribution} (\textsc{SubSearch},~\citep{bianco2025subsearchrobustestimationoutlier}): finding \(S\) by optimizing \(c(S) \coloneqq \lVert A_S - \hat{Q}(S) \rVert_{\textup{op}}\) via Simulated Annealing
        \item \textbf{Contribution}: {\color{myblue}\url{github.com/leobianco/robust_estim_sbm}}
    \end{itemize}
\end{frame}

% \begin{frame}{Results}
%     \begin{algorithm}[H]
%         \caption{\textsc{SubSearch}}
%         \begin{algorithmic}
%             \Require Graph \(A\), subgraph size \((1-\gamma)n\)
%             \State \(S_{\text{best}} \gets\) random connected subgraph of size \((1-\gamma)n\)
%             \State Initialize temperature \(T_0\)
%             \For{\(t = 1, \dotsc, t_{\text{max}}\)}
%                 \For{each Markov chain iteration}
%                     \State \(S_{\text{candidate}} \gets \texttt{neighbor}(S_{\text{current}})\)
%                     \State \(\Delta \gets c(S_{\text{current}}) - c(S_{\text{candidate}})\)
%                     \State Accept \(S_{\text{candidate}}\) with probability \(\min(1, e^{\Delta / T_t})\)
%                     \State Update \(S_{\text{best}}\) if improved
%                 \EndFor
%                 \State \(T_{t+1} \gets c \cdot T_t\)
%             \EndFor
%             \State \Return \(S_{\text{best}}\)
%         \end{algorithmic}
%     \end{algorithm}
% \end{frame}

% \begin{frame}{\textsc{SubSearch}: Subgraph Search via Simulated Annealing}
% We explore the space \(\mathcal{S}\) of all subgraphs \(S \subset G\) of size \((1-\gamma)n\), 
% seeking a minimizer of the objective
% \[
% c(S) = \|A_S - \hat{Q}(S)\|_{\textup{op}}.
% \]
% \vspace{1em}
% \begin{itemize}
%     \item[\textbf{Step 1.}] Initialize the search at a high ``temperature'' \(T_0\).
%     \pause
%     \item[\textbf{Step 2.}] Propose a new subgraph by swapping one node 
%     \(i \in S_{\text{current}}\) with a random node \(j \notin S_{\text{current}}\).
%     \pause
%     \item[\textbf{Step 3.}] Evaluate the change in cost 
%     \(\Delta = c(S_{\text{current}}) - c(S_{\text{candidate}})\).
%     Accept the candidate with probability
%     \[
%     \min\!\left(1,\; \exp\!\left(\Delta / T_t\right)\right).
%     \]
%     \pause
%     \item[\textbf{Step 4.}] Gradually cool the system using 
%     \(T_{t+1} = c\, T_t\), where the cooling rate \(c \approx 1\).
% \end{itemize}
% \end{frame}

\begin{frame}{\textsc{SubSearch}: Subgraph Search via Simulated Annealing}

\begin{columns}[T, totalwidth=\textwidth]
% -------------------------------- LEFT COLUMN --------------------------------
\begin{column}{0.55\textwidth}
    \small

Explore the space \(\mathcal{S}\) of subgraphs \(S \subset G\) of size \((1-\gamma)n\),
to minimize \(c(S) = \|A_S - \hat{Q}(S)\|_{\textup{op}}\)

\vspace{0.5em}

\begin{itemize}
    \uncover<2->{%
        \item[\color{lightblue}\large\ding{172}] {\color{lightblue}Initialize:}
        random subgraph \(S_0\), high temperature \(T_0\)
    }

    \uncover<3->{%
        \item[\color{lightblue}\large\ding{173}] {\color{lightblue}Propose
        \(S_{\text{candidate}}\):} swap 
        \(i \in S_{\text{current}}\) with \(j \notin S_{\text{current}}\)
    }

    \uncover<4->{%
        \item[\color{lightblue}\large\ding{174}] {\color{lightblue}Accept or reject:}
        compute \(\Delta = c(S_{\text{current}}) - c(S_{\text{candidate}})\),\\
        accept with probability \(\min\!\left(1, \exp(\Delta / T_t)\right)\)
    }

    \uncover<5->{%
        \item[\color{lightblue}\large\ding{175}] {\color{lightblue}Cool down:}
        \(T_{t+1} = c\,T_t\), \(c \approx 1\)
    }
    \uncover<6->{}
\end{itemize}

\end{column}

% -------------------------------- RIGHT COLUMN --------------------------------
\begin{column}{0.45\textwidth}
\centering
\vbox{\vspace{5em} % optional top padding to stabilize alignment
\only<1>{\includegraphics[width=\linewidth,keepaspectratio]{figures/part1/sbm_outliers.png}}
\only<2>{\includegraphics[width=\linewidth,keepaspectratio]{figures/part1/sbm_outliers_init.png}}
\only<3>{\includegraphics[width=\linewidth,keepaspectratio]{figures/part1/sbm_outliers_candidate.png}}
\only<4>{\includegraphics[width=\linewidth,keepaspectratio]{figures/part1/sbm_outliers_current.png}}
\only<5>{\includegraphics[width=\linewidth,keepaspectratio]{figures/part1/sbm_outliers_current.png}}
\only<6>{\includegraphics[width=\linewidth,keepaspectratio]{figures/part1/sbm_outliers_ideal.png}}
}
\end{column}
\end{columns}
\end{frame}

\begin{frame}{Results}
    \begin{figure}
        \centering
        \includegraphics[width=\linewidth]{figures/part1/jazz_sc_graph.png}
        % \caption{Using our method, outliers are removed and a partition correlated with NYC, Chicago, and ``Other Cities'' is found.}
    \end{figure} 
\end{frame}

\begin{frame}{Results}
    \addtocounter{framenumber}{-1}
    \begin{figure}
        \centering
        \includegraphics[width=\linewidth]{figures/part1/jazz_SA_graph.png}
        % \caption{Using our method, outliers are removed and a partition correlated with NYC, Chicago, and ``Other Cities'' is found.}
    \end{figure} 
\end{frame}

\begin{frame}{Results}
    Parameters: \(n = 200\), \(K = 2\), \(\gamma = 0.3\).
    \begin{figure}
        \centering
        \begin{subfigure}{\textwidth}
            \centering
            \includegraphics[height=0.4\textheight]{figures/part1/ideal_SA_horizontal.pdf}
            \label{fig:simulated_annealing_single_run}
        \end{subfigure}
        \vspace{0.1em}
        \begin{subfigure}{\textwidth}
            \centering
            \includegraphics[height=0.45\textheight]{figures/part1/simulated_annealing_true_outliers.png}
            \label{fig:simulated_annealing_true_outliers}
        \end{subfigure}
    \end{figure}
\end{frame}

\begin{frame}{Results}
    Parameters: \(n = 200\), \(K = 2\), \(\gamma = 0.3\).
    \begin{figure}
        \begin{subfigure}{\textwidth}
            \centering
            \includegraphics[height=0.4\textheight]{figures/part1/filtering_horizontal.pdf}
            \label{fig:filtering_horizontal_1}
        \end{subfigure}
        \begin{subfigure}{\textwidth}
            \centering
            \includegraphics[height=0.45\textheight]{figures/part1/filtering_true_outliers.png}
            \label{fig:filtering_graph}
        \end{subfigure}
    \end{figure}
\end{frame}

\begin{frame}{Results}%[allowframebreaks, noframenumbering]
    \begin{figure}
        \centering
        \includegraphics[width=\textwidth]{figures/part1/multirun_gamma.pdf}
        % \caption{Estimation error of different methods as the amount of perturbation increases.}
        \label{fig:error_vs_gamma_best_results}
    \end{figure}
\end{frame}

\begin{frame}{Discussion}
    \begin{itemize}
        \item \textbf{Main takeaway}: ``exploring'' the space of subgraphs \(\Rightarrow\) find subgraphs avoiding outliers
        \item Perspective \# 1: can we rigorously prove robustness?
        \item Perspective \# 2: can we provide faster rates?
    \end{itemize}
    
\end{frame}